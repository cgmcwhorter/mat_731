% !TEX root = ../../rings_mods.tex
\section{Free Modules, Projective Modules, and Injective Modules}
\subsection{Free Modules}

The `simplest' type of modules are free modules. 

\begin{dfn}[Free Module]
A left $R$-module $F$ is a free left $R$-module if $F$ is isomorphic to a direct sum of copies of $R$; that is, there is a (possibly infinite) index set $\mathcal{B}$ such that $F= \bigoplus_{i \in \mathcal{B}} R_i$, where $R_i=\langle b_i \rangle \cong R$ for all $i \in \mathcal{B}$. We call $\mathcal{B}$ a basis for $F$.
\end{dfn}

The term `free' refers to the fact that the basis elements have no $R$-linear relations, i.e. there are no collections $\{b_s\}$, $\{r_s\}$ such that $\sum_j r_jb_j=0$. A free $\Z$-module is called a free abelian group. Every ring $R$, when considered as a left module over itself, is a free $R$-module. Despite having a rigid structure, free modules have a rich theory---think of Linear Algebra. In fact in the case where $R=k$ is a field, this is precisely Linear Algebra. These modules are also very ubiquitous, as the following proposition shows.

\begin{prop} \label{prop:setbasis}
Let $R$ be a ring. Given any set $B$, there exists a free $R$-module $F$ with basis $B$.
\end{prop}

\pf The set of functions $R^B=\{ \phi: B \to R\}$ is a left $R$-module, where for all $b \in B$ and $r \in R$, define $\phi+\psi: B \to R$ via $b \mapsto \phi(b)+\psi(b)$ and $r\phi: B \to R$ via $r \cdot \phi(b)$. Define the function $\mu_b$ as
	\[
	\mu_b(a)= 
	\begin{cases}
	1, & \text{if } a=b \\
	0, & \text{if } a \neq b
	\end{cases}
	\]
Denoting $\mu_b$ by $b$, $R^B$ is the direct product $\prod_{b \in B} \langle b \rangle$. Now $R \cong \langle b \rangle$ via the map $r \mapsto r\mu_b$. Then the submodule $F$ of $R^B$, generated by $B$, is a direct sum of copies of $R$. But then $F$ is a free left $R$-module with basis $B$. \qed \\


Free modules also have strong properties on their maps as well. In Linear Algebra, one has the notion of extending maps by linearity. One has a similar result for free modules.

\begin{prop} \label{prop:freemap}
Let $R$ be a ring and let $F$ be a free left $R$-module on a basis $B$. If $M$ is a left $R$-module and $f: B \to M$ is a function, there exists a unique $R$-map $\tilde{f}: F \to M$ with $\tilde{f}\mu=f$, where $\mu: B \to F$ is the inclusion map; that is, $\tilde{f}(b)=f(b)$ for all $b \in B$, i.e. $\tilde{f}$ extends $f$.
\end{prop}

\pf Every $v \in F$ has a unique expression of the form $v= \sum_{b \in B} r_bb$, where $r_b \in R$ and $r_b=0$ for almost all $b$. Therefore, there is a well defined function $\tilde{f}: F \to M$ given by $v \mapsto \sum_{b \in B} r_b f(b)$. It is routine to verify that $\tilde{f}$ extends $f$. Then if $s \in R$, $sv= \sum sr_bb$. If $v'= \sum r_b'b$, then $v+v'=\sum(r_b+r_b')x$. But then $\tilde{f}$ is an $R$-map. Since, $F= \langle B \rangle$, $\tilde{f}$ is the unique map extending $f$. [One can easily verify that two $R$-maps agreeing on a generating set are equal.] \qed \\

Free modules share further parallels to Linear Algebra, as the reader can easily verify. 

\begin{prop}
Let $R$ be a nonzero commutative ring.
	\begin{enumerate}[(i)]
	\item Any two bases of a free $R$-module $F$ have the same cardinality.
	\item Free $R$-modules $F$ and $F'$ are isomorphic if and only if there are bases having the same cardinality.
	\item If $n$ and $m$ are integers, then $R^n \cong R^m$ if and only if $n=m$. 
	\end{enumerate}
\end{prop}

If a ring $R$ has the property that $R^n \cong R^m$, where $n$ and $m$ are integers, that $n=m$ is said to have IBN (invariant basis number). If $R$ has IBN, the number of elements in a basis of a free $R$-module $F$ is called the rank of $F$, denoted $\rank F$. By the work above, if $R$ has IBN and $F$ is a finitely generated free left $R$-module, then every two bases of $F$ have the same number of elements. All nonzero commutative rings $R$ have IBN. The proof of this generalizes to show that any noncommutative ring $R$ with a two-sided ideal $I$ for which $R/I$ is a division ring, e.g. every local ring, has IBN. Furthermore, every division ring and noetherian ring has IBN. For an example where $R$ does not have IBN, take $R=\End_k(V)$, where $k$ is a field and $V$ is an infinite dimensional vector space over $k$. [For this, consider maps $\phi: V \to V \oplus V$ and $V \oplus V \to V$.] Finally, every module arises from a free module in some way.


\begin{thm}
Every left $R$-module $M$ is the quotient of a free left $R$-module $F$. Moreover, $M$ is finitely generated if and only if $F$ can be chosen to be finitely generated. 
\end{thm}

\pf Choose a generating set $X$ of $M$. Let $F$ be a free module on basis $\{b_x \colon x \in X\}$ (this makes use of Proposition~\ref{prop:setbasis}). By Proposition~\ref{prop:freemap}, there exists an $R$-map $g: F \to M$, where $g(b_x)=x$ for all $x \in X$. Clearly, $g$ is a surjection as $\im g$ is a submodule of $M$ containing $X$. But then $F/\ker g \cong M$. If $M$ is finitely generated, then there is a finite generating set $X$ and the free module $F$ constructed above is finitely generated. The converse is immediate since the image of a finitely generated module is finitely generated. \qed \\

This theorem implies that there given any $R$-module, there is always a free module which surjectively maps onto it. We will often use this theorem without mention. One could always obtain this from Proposition~\ref{prop:setbasis}. 


\subsection{Projective Modules}


\begin{dfn}[Projective Module]
An $R$-module $_R P$ is projective if for all homomorphisms $B \ma{g} C \ma{} 0$ and maps $f: P \to C$, there exists a lift of $f$ to $B$; that is, there exists a homomorphism $h: P \to B$ with $gh= f$.
	\[
	\begin{tikzcd}
	& P \arrow{d}{f} \arrow[dotted,swap]{dl}{h} & \\
	B \arrow{r}{g} & C \arrow{r} & 0
	\end{tikzcd}
	\]
\end{dfn}


\begin{lem}
Every free module is projective.
\end{lem}

\pf Assume that $F$ is free on a basis $\{t_\alpha\}_{\alpha \in \mathcal{I}}$. 
	\[
	\begin{tikzcd}
	& F \arrow{d}{f} \arrow[dotted,swap]{dl}{h} & \\
	B \arrow{r}{g} & C \arrow{r} & 0
	\end{tikzcd}
	\]
We examine the image of the basis under $f$: $\{f(t_\alpha)\}_{\alpha \in \mathcal{I}} \subseteq C$. As $g$ is onto, for all $\alpha \in \mathcal{I}$, we have $f(t_\alpha)= g(b_\alpha)$ for some $b_\alpha \in B$. Let $h: F \to B$ be the unique homomorphism with $h(t_\alpha)= b_\alpha$, extending by linearity. Therefore, we have
	\[
	g\left( h\left( \sum_i r_i t_i \right)\right) =  \sum_i g(h(r_it_i))= \sum_i r_i g(h(t_i))= \sum_i r_i g(b_i)= \sum_i r_i f(t_i)=  f\left(\sum_i r_it_i\right).
	\]
But then $gh=f$ so that $F$ is projective. \qed \\


In fact in some sense, projective modules are `close' to being free modules in that they have a dual basis, see Lemma~\ref{lem:dualbasis}. There are also many equivalent definitions for a projective module, each useful in various situations. 


\begin{prop}\label{sumprojproj}
The following are equivalent for a module $P$:
	\begin{enumerate}[(i)]
	\item $P$ is a projective module.
	\item For all $X \ma{g} P$ with $g$ onto, the mapping splits. That is, there exists $h: P \to X$ with $gh=1_P$.
	\item $P$ is isomorphic to a direct summand of a free module.
	\item $\Hom(P,-)$ is exact; that is, if
		\[
		0 \ma{} A \ma{f} B \ma{g} C \ma{} 0
		\]
is any exact sequence, then
		\[
		0 \ma{} \Hom(P,A) \ma{f_*} \Hom(P,B) \ma{g_*} \Hom(P,C) \ma{} 0
		\]
is exact.
	\end{enumerate}
\end{prop}

\pf $(i)\to(ii)$: Suppose $g: X \to P$ is onto. Consider the identity map $1_P: P \to P$.
	\[
	\begin{tikzcd}
	& P \arrow{d}{1} \arrow[dotted,swap]{dl}{h} &  \\
	X \arrow{r}{g} & P \arrow{r} & 0 
	\end{tikzcd}
	\]
Since $P$ is projective, there exists a lift of $1_P$ to $X$, i.e. a map $h: P \to X$ such that $hg=1_P$. 

$(ii)\to(iii)$: Let $F$ be a free module mapping surjectively onto $P$ via a map $g: F \to P$. There is an exact sequence
	\[
	0 \ma{} \ker g \ma{\iota} F \ma{g} P \ma{} 0.
	\]
By assumption, there exists a map $h: P \to F$ such that $gh=1_P$. But then by Theorem~\ref{thm:splitseq}, $P$ is a direct summand of $F$.

$(iii)\to(iv)$: Let 
	\[
	0 \ma{} A \ma{f} B \ma{g} C \ma{} 0 
	\]
be an exact sequence. By Proposition~\ref{prop:homexact}, we already know that $\Hom(P,-)$ is left exact. We need only prove exactness on the right, i.e. exactness at $\Hom(P,C)$. This is proving that $g^*$ is surjective. Let $\phi \in \Hom(P,C)$. By assumption, $P$ is a direct summand of $F$, say $F= A \oplus P$. We have a diagram
	\[
	\begin{tikzcd}
	& F \arrow{d}{\pi} \arrow[xshift=-1ex,dotted,swap]{ddl}{\Psi} & \\
	& P \arrow{d}{\phi} \arrow[xshift=0.5ex,pos=0.0,swap,dotted]{dl}{\Psi\big|_P} & \\
	B \arrow[swap]{r}{g} & C \arrow{r} & 0 
	\end{tikzcd}
	\]
There is the canonical surjection from $F$ onto $P$, $\pi: F \to P$. Then $\phi\pi: F \to C$ is an $R$-map. By Proposition~\ref{prop:freemap}, there is a map $\Psi: F \to B$ such that $g\Psi=\phi\pi$. Now $\pi\big|_P=1_P$. Then $g\Psi\big|_P=\phi\pi\big|_P=\phi1_P=\phi$. Define $\tilde{\Psi}:=\Psi\big|_P: P \to B$. Then $\tilde{\Psi} \in \Hom(B,C)$ with $g_*(\tilde{\Psi})=g\tilde{\Psi}=g\Psi\big|_P=\phi \in \Hom(P,C)$ so that $g_*$ is surjective. 

$(iv)\to(i)$: Suppose that $g: B \to C$ is a surjection and $\phi: P \to C$ is an $R$-map. We have an exact sequence and diagram
	\[
	\begin{tikzcd}
	& & & P \arrow{d}{\phi} \arrow[swap,dotted]{dl}{h} & \\
	0 \arrow{r} & \ker g \arrow{r}{\iota} & B \arrow{r}{g} & C \arrow{r} & 0 
	\end{tikzcd}
	\]
Now there is an exact sequence
	\[
	0 \ma{} \Hom(P,\ker g) \ma{\iota_*} \Hom(P,B) \ma{g_*} \Hom(P,C) \ma{} 0
	\]
Since $g_*$ is surjective, there exists a map $h \in \Hom(P,B)$, i.e. an $R$-map $h: P \to B$, such that $g_*(h)=\phi$. But $\phi=g_*(h)=gh$ so that $P$ is projective. \qed \\


 The reader should try to prove direct equivalences between all the equivalent conditions of the previous proposition as an exercise. Furthermore, these equivalences allow us to prove the Dual Basis Lemma.
 

\begin{lem}[Dual Basis Lemma] \label{lem:dualbasis}
An $R$-module $P$ is projective if and only if there exists a family of elements $\{a_i\}_{i \in \mathcal{I}} \subseteq P$ and linear functions $\{f_i\}_{i \in \mathcal{I}} \subseteq P^*=\Hom_R(P,R)$ such that for any $a \in P$, $f_i(a)=0$ for almost all $i$ and $a= \sum_i a_if_i(a)$. 
\end{lem}

\pf Suppose that $P$ is projective. Fix an epimorphism $g$ from a free module $F= \bigoplus Re_i$ onto $P$. But as $P$ is projective, $g$ has a splitting $h: P \to F$, which can be expressed as
	\[
	h(a)= \sum_i f_i(a)e_i.
	\]
The $f_i$ are $R$-linear and $f_i(a)=0$ for almost all $i$. Therefore, $f_i \in P^*$. But then
	\[
	a= gh(a) = \sum f_i(a)a_i,
	\]
where $a_i:= g(e_i) \in P$. 

Now suppose that the $a_i,f_i$ exist as in the statement of the lemma. Define $F: \bigoplus Re_i$ and an epimorphism $g: F \to P$ given by $g(e_i)=a_i$ for all $i \in \mathcal{I}$. Define also a map $h: P \to F$ via $a \mapsto \sum f_i(a)e_i$. One routinely verifies that $h$ is an $R$-map. It is also routine to verify that $h$ is a splitting for $g$. But then $P$ is isomorphic to a direct summand of $F$. Therefore, $P$ is projective. \qed \\


\begin{rem}
The pairings $\{(a_i,f_i)\}_{i \in \mathcal{I}}$ are often referred to as ``a pair of dual bases.'' Note that the $a_i$ are only a generating set for $P$ and are not necessarily a basis for $P$.
\end{rem}


\begin{prop} \hfill
\begin{enumerate}[(i)]
\item Every direct summand of a projective module is itself projective.
\item Every direct sum of projective modules is projective.
\end{enumerate}
\end{prop}

\pf
\begin{enumerate}[(i)] 
\item A module is projective if and only it is a direct summand of a free module. But then any module that is a summand of a projective module is a summand of a free module, and hence is projective.
\item Let $\{P_I\}_{i \in \mathcal{I}}$ be a family of projective modules. For all $i \in \mathcal{I}$, there exists a free module $F_i$ such that $F_i= P_i \oplus Q_i$ for some $Q_i \subseteq F_i$. Now $\bigoplus_{i \in \mathcal{I}} F_i$ is free (a basis being the union of the bases for the $F_i$), and
	\[
	\bigoplus_{i \in \mathcal{I}} F_i = \bigoplus_{i \in \mathcal{I}} (P_i \oplus Q_i)= \bigoplus_{i \in \mathcal{I}} P_i \oplus \bigoplus_{i \in \mathcal{I}} Q_i. 
	\]
But then $\bigoplus_{i \in \mathcal{I}} P_i$ is a summand of a free module, hence free. \qed \\
\end{enumerate}

In fact, the last statement in the above proposition is an if and only if.


\begin{prop}
Let $\{P_i\}_{i \in \mathcal{I}}$ be a family of modules. Then $\bigoplus_{i \in \mathcal{I}}$ is projective if and only if $P_i$ is projective. 
\end{prop}

\pf The reverse direction was shown in the previous proposition. We need only show the forward direction. Suppose that $\bigoplus_{i \in \mathcal{I}}$ is projective. Let $g: B \to C$ be a surjection and let $f: \bigoplus_{i \in \mathcal{I}} \to C$ be an $R$-map. Consider the following diagram:
	\[
	\begin{tikzcd}
	B \arrow{r}{g} & C \arrow{r} & 0 \\
	& \bigoplus_{i \in \mathcal{I}} P_i \arrow[xshift=1ex,dotted]{ul}{h} \arrow{u}{f} \\
	& P_i \arrow{u}{\iota_i} \arrow[dotted]{uul}{h_i}
	\end{tikzcd}
	\]
where $\iota_i$ is the canonical injection. Since $\bigoplus_{i \in \mathcal{I}}$ is projective, there exists a lift $h$ of $f$ to $B$. Define $h_i: P_i \to B$ via $h_i:= h\iota_i$. Clearly, $h_i$ is an $R$-map. We have $f=hg$ so that $f\big|_{P_i}=h\big|_{P_i}g=h\iota_ig=h_ig$. But then $P_i$ is projective. \qed \\


\begin{rem}
If $R$ is a local principal ideal domain, then $R$ is free as an $R$-module.  
\end{rem}


\begin{ex}
Not all projective modules are free. Let
	\[
	R=\begin{pmatrix} \Q & 0 & 0 \\ \Q & \Q & 0 \\ \Q & \Q & \Q \end{pmatrix}
	\]
That is, let $R$ be the set of lower triangular matrices. We know $\dim_\Q R=6$. So if $F$ is a finite dimensional free module, then $\dim F$ is a multiple of 6.
	\[
	P=\begin{pmatrix} 0 & 0 & 0 \\ 0 & \Q & 0 \\ 0 & \Q & 0 \end{pmatrix} \subseteq R
	\]
As $P$ is a submodule (in fact a left ideal of $R$), then as a left module $\dim_R P=2$ so that $P$ is not free. \xqed
\end{ex}


\begin{ex}
Let $R=\Z/6\Z$. Note that $R=\Z/6\Z=\Z/2\Z \oplus \Z/3\Z$. Let $I=\Z/2\Z$ and $J=\Z/3\Z$. Now $R$ is free as an $R$-module. Since $I,J$ are direct summands of $R$, $I$ and $J$ are projective $\Z/6\Z$-modules. However, neither $I$ nor $J$ are free as a (finitely generated) free $\Z/6\Z$-module must be a direct sum of $n$-copies of $\Z/6\Z$, and so they must have $6^n$ elements. However, both $I$ and $J$ are too small for this to be the case. \xqed
\end{ex}


\begin{prop}
Suppose that the following diagram is commutative with exact rows
	\[
	\begin{tikzcd}
	A \arrow{r}{f} \arrow{d}{\alpha} & B \arrow{r}{g} \arrow{d}{\beta} & C \arrow{r} & 0 \\
	A' \arrow{r}{f'} & B' \arrow{r}{g'} & C' \arrow{r} & 0
	\end{tikzcd}
	\]
Then there exists a unique map $h: C \to C'$ making the diagram commute. Moreover, $h$ is an isomorphism if $\alpha$ and $\beta$ are isomorphisms. 
\end{prop}

\pf Assume we have the following commutative diagram with exact rows.
	\[
	\begin{tikzcd}
	A \arrow{r}{f} \arrow{d}{\alpha} & B \arrow{r}{g} \arrow{d}{\beta} & C \arrow{r} \arrow[dotted]{d}{h} & 0 \\
	A' \arrow{r}{f'} & B' \arrow{r}{g'} & C' \arrow{r} & 0
	\end{tikzcd}
	\]




































Then there exists a unique $h: C \rightarrow C'$ such that $hg=g' \beta$. We show this first. Let $x \in C$. As $g$ is onto, there is a $b \in B$ such that $x=g(b)$. 
\[
\begin{tikzcd}
b \arrow{r}{g} \arrow{d}{\beta} & x=g(b) \\
\beta(b) \arrow{r} & g'\beta(b)
\end{tikzcd}
\]
We ``try'' the map $h(x) \defeq g'\beta(b)$. We need check that this map is well defined. Let $b_1 \in B$ such that $g(b_1)=g(b)=x$. We have
\[
\begin{tikzcd}
b_1 \arrow{dr} & \\
b \arrow{r}{g} \arrow{d}{\beta} & x=g(b) \\
\beta(b) \arrow{r} & g'\beta(b)
\end{tikzcd}
\]
We need check that $g'(\beta(b))=g'(\beta(b;))$. Note that $g(b_1-b)=0$ so that $b_1-b=f(a)$ for some $a \in A$. Then $\beta(b_1-b)=\beta(f(a))=f'(\alpha(a))$. Moreover, $g'(\beta(b_1-b))=g'(f'(\alpha(a)))=0$, because of exactness. So we know that $g'(\beta(b_1))=g'(\beta(b))$ so that the map is well defined. By construction, it commutes the diagram. 

However, there is an easier way of demonstrating this. 
\[
\begin{tikzcd}
A \arrow{r}{f} \arrow{d}{\alpha} & B \arrow{r}{g} \arrow{d}{\beta} & C \arrow{r} \arrow[dotted]{d}{h} & 0 \\
A' \arrow{r}{f'} & B' \arrow{r}{g'} & C' \arrow{r} & 0
\end{tikzcd}
\]
Note we have $B \stackrel{g}{\longrightarrow} C$ and look at the cokernel of $f$. We have $g'\beta f=g'f'\alpha=0$. Then simply use the Universal Property of the Cokernel. 
\[
\begin{tikzcd}
A \arrow{r}{f} & B \arrow{d}{g'\beta} \arrow{r}{g} & C \arrow[dotted]{dl}{\exists!h} \\
& C' & 
\end{tikzcd}
\]
We have yet to show that $h$ is unique and a homomorphism. We show this by showing that for any commutative diagram with exact rows, as below,
\[
\begin{tikzcd}
0 \arrow{r} & A \arrow{r}{f} \arrow[dotted]{d}{h} & B \arrow{r}{g} \arrow{d}{\beta} & C \arrow{d}{\gamma} \\
0 \arrow{r} & A' \arrow{r}{f'} & B' \arrow{r}{g'}  & C'
\end{tikzcd}
\]
there exists a unique $h: A \rightarrow A'$ with $f'h=\beta f$. The idea is the same as above but we use the Universal Property of the Kernel. We have $f': A' \rightarrow B'$ is the kernel of $g'$.
\[
\begin{tikzcd}
\; & \; & A \arrow[dotted]{dl}{h} \arrow{d}{\beta f} & \\
0 \arrow{r} & A' \arrow{r}{f'} & B' \arrow{r}{g'} & C'
\end{tikzcd}
\]
We know also that $g'\beta f=0$. But then the Universal Property of the Kernel says there exists a unique $h: A \rightarrow A'$ with $f'h=\beta f$. \qed \\


\subsection{Injective Modules}


Let $f: A \rightarrow B$ be a homomorphism and let $M$ be another $R$-module. We have an induced map of abelian groups
\[
\Hom_R(B,M) \longrightarrow_{\Hom_R(f,M)=f_*^M} \Hom_R(A,M)
\]
given by $f_*^M(\beta)=\beta f$. 

\begin{prop}
Let $0 \longrightarrow A \ma{f} B \ma{g} C \longrightarrow 0$ be a short exact sequence. Then the sequence
\[
0 \longrightarrow \Hom(C,M) \ma{g_*^M} \Hom(B,M) \ma{f_*^M} \Hom(A,M)
\]
is exact.
\end{prop}

Proof: Exercise \\


Note that $f_*^M$ is onto if for any $h: A \rightarrow M$, there is a $p: B \rightarrow M$ with $pf=h$. 
\[
\begin{tikzcd}
0 \arrow{r} & A \arrow{r}{f} \arrow{d}{h} & B \arrow[dotted]{dl}{\exists p}\\
 & M & 
\end{tikzcd}
\]
Furthermore, we know that if $0 \longrightarrow A \ma{f} B \ma{g} C \longrightarrow 0$ is a split exact sequence then for all $M$ the sequence 
\[
0 \longrightarrow \Hom_R(C,M) \longrightarrow \Hom_R(B,M) \longrightarrow \Hom_R(A,M) \longrightarrow 0
\]
is exact. 

\begin{dfn}[Injective Module]
An $R$-module $I$ is injective if whenever we have
\[
\begin{tikzcd}
0 \arrow{r} & A \arrow{r}{f} \arrow{d}{h} & B \arrow[dotted]{dl}{g} \\
& I & 
\end{tikzcd}
\]
$h$ can be ``extended'' to $B$. That is, there exists $g: B \rightarrow I$ with $gf=h$. 
\end{dfn}

First note that $_R I$ is injective if and only if for all short exact sequences
\[
0 \longrightarrow A \ma{f} B \ma{g} C \longrightarrow 0
\]
the following sequence is exact
\[
0 \longrightarrow \Hom(C,M) \ma{g_*^M} \Hom(B,M) \ma{f_*^M} \Hom(A,M) \longrightarrow 0
\]

\begin{prop}
$_R I$ is injective if and only if for all monomorphisms $0 \longrightarrow I \ma{f} X$ splits, i.e. there exists $p: X \rightarrow I$ with $pf=\text{id}_I$. So $I$ is isomorphic to a direct summand of $X$.
\end{prop}


\subsection{Baer's Criterion} 


\begin{thm}[Baer's Criterion]
Let $R$ be a ring and $E$ a left $R$-module. Then $_R E$ is injective if and only if for all left ideals $_R I$ of $R$ and 
\[
\begin{tikzcd}
0 \arrow{r} & I \arrow[hook]{r}{\text{incl}} \arrow{d}{f} & R \arrow[dotted]{dl}{s} \\
& E & 
\end{tikzcd}
\]
then $f$ can be extended to $R$, i.e. there exists a homomorphism $s: R \rightarrow E$ such that $s|_I=f$. 
\end{thm}

Proof: The forward direction is trivial as when $E$ is injective the result is trivial. Now assume the converse. Let the following the homomorphisms of $R$-modules
\[
\begin{tikzcd}
0 \arrow{r} & A \arrow{r} \arrow{d}{f} & B  \\
& E & 
\end{tikzcd}
\]
Without loss of generality, assume that $A \ma{\text{incl}} B$. (Why?) Then we have
\[
\begin{tikzcd}
0 \arrow{r} & A \arrow[hook]{r} \arrow{d}{f} & B  \\
& E & 
\end{tikzcd}
\]
Let $S=\{(A',f') \;|\; A \subseteq A' \subseteq B, f':A' \rightarrow E \text{ and extends }f\}$. 
\[
\begin{tikzcd}
A \arrow[hook]{r} \arrow{d}{f} & A' \arrow[hook]{r} \arrow{dl}{f'} & B \\
E
\end{tikzcd}
\]
We know that $S \neq \emptyset$ as the pair $(A,f) \in S$. We put an ordering on $S$ given by $(A',f') \leq (A'',f'')$ if 
\[
\begin{tikzcd}
A \arrow{d}{f} & \subseteq & A' \arrow{dll}{f'} & \subseteq & A'' \arrow{dllll}{f''} & \subseteq & B \\
E & & & & & & 
\end{tikzcd}
\]
This is an ordering. (Why?) We claim that $S$ has a maximal element. Pick a chain $\{(A_i,f_i)\}_i$ in $S$. We look at $\left( \bigcup A_i,\overline{f}\right)$, where $\overline{f}: \bigcup A_i \rightarrow E$ defined by $\overline{f}(a_i)=f_i(a_i)$. This is an upper bound for the chain. Then Zorn's Lemma says that there exists $(A^*,f^*)$ which is a maximal element of $S$.
\[
\begin{tikzcd}
0 \arrow{r} & A \arrow[hook]{r} \arrow{d}{f} & A^* \arrow[hook]{r} \arrow{dl}{f^*} & B \\
& E & & 
\end{tikzcd}
\]
We want to show that $A^*=B$. Suppose that this is not the case. Then there is a $b \in B \setminus A^*$. Then
\[
\begin{tikzcd}
0 \arrow[hook]{r} & A \arrow[hook]{r} & A^* \arrow[hook]{r} \arrow{d}{f^*} & A^*+\langle b \rangle=\overline{A} \\
& & E & 
\end{tikzcd}
\]
Let $I=\{r \in R \;|\; rb \in A^*\}$. We know that $0 \in I$ so that $I \neq \emptyset$. It is trivial to show that $I$ is a left ideal of $R$. Then
\[
\begin{tikzcd}
0 \arrow{r} & I \arrow{d}{j} \arrow[hook]{r} & R \arrow[dotted]{dl}{\exists \overline{j}} \\
& E & 
\end{tikzcd}
\]
where $j(r) \defeq f^*(rb)$. The map $j$ is a $R$-module homomorphism where $\overline{j}|_I =j$. We look at $\overline{j}(1) \in E$. We construct $\overline{f}: \overline{A} \rightarrow B$ so that $\overline{f}|_{A^*} =f^*$. Then we have $\overline{f}|_A=f$. So we have $(\overline{A},\overline{f}) >(A^*,f^*)$ as $\overline{A} \supsetneq A^*$. This contradicts our ordering from above; that is, this contradicts the maximality of $(A^*,f^*)$.

Define $\overline{f}(a^*+rb)=f^*(a^*)+r\overline{j}(1)$. We claim that $\overline{f}$ is well defined and is a map of $R$-modules. We first show the map is well defined. Let $a_1^*+r_1b = a_2^*+r_2b$. Then we have $a_1^*-a_2^*=(r_2-r_1)b$. Then $r_2-r_1 \in I$.
\[
\begin{split}
f^*(a_1^*-a_2^*)=f^*((r_2-r_1)b)=j(r_2-r_1)=\overline{j}(r_2-r_1)=\overline{j}(r_2)-\overline{j}(r_1)
\end{split}
\]
But $f^*(a_1^*)-f^*(a_2^*)=f^*(a_1^*-a_2^*)$ so that $f^*(a_1^*)+\overline{j}(r_1)=f^*(a_2^*)+\overline{j}(r_2)=\overline{f}(a_2^*+r_2b)$. But $\overline{f}(a_1^*+r_1b)=f^*(a_1^*)+\overline{j}(r_1)$. Therefore, the map is well defined. It remains to show that $\overline{f}$ is a $R$-module homomorphism. (Exercise) \qed \\

Now recall that if $\{P_i\}_{i \in \mathcal{I}}$ is a family of modules, then each $P_i$ is projective if and only if $\bigoplus_{i \in \mathcal{I}} P_i$ is projective. We have a similar result for injective modules.

\begin{prop}
Let $\{E_i\}_{i \in \mathcal{I}}$ be a family of $R$-modules. Then each $E_i$ is injective if and only if $\prod_{i \in \mathcal{I}} E_i$ is injective. 
\end{prop}

Proof: Let each $E_i$ be injective. 
\[
\begin{tikzcd}
0 \arrow{r} & A \arrow{d}{f} \arrow{r}{j} & B \arrow[dotted]{dl}{\exists g} \arrow[dotted]{ddl}{g_i} \\
& \prod E_i \arrow{d}{\pi_i} & \\
& E_i &
\end{tikzcd}
\]
We need find a map $g: B \rightarrow \prod E_i$. As the $E_i$ are injective, there is a map $g_i: B \rightarrow E_i$. Let $g=\prod g_i$. That is, let $g(b)=\big(g_i(b)\big)_{i \in \mathcal{I}}$. This map works. (Why?)

Now assume $\prod E_i$ is injective. 
\[
\begin{tikzcd}
0 \arrow{r} & A \arrow{d}{f} \arrow{r}{j} & B \arrow[dotted]{dl}{\exists g_i} \arrow[dotted]{ddl}{\exists f} \\
&  E_i \arrow[hook]{d} & \\
&\prod E_i &
\end{tikzcd}
\]
Let $k_i(x)=(0,0,\cdots,x,0,0,\cdots,0)$, where the $x$ occurs in the $i$th position. As the $\prod E_i$ is injective, there exists a module homomorphism $f: B \rightarrow \prod E_i$. Note that there is also $\pi_i: \prod E_i \rightarrow E_i$, where $\pi_i\big((x_i)_{i \in \mathcal{I}} \big) \defeq x_i$ and $\pi_i k_i=1_{E_i}$. Let $g_i=\pi_i f$. Then one easily checks $\pi_i f j=\underbrace{\pi_i k_i}_{\text{id}} f_i=f_i$. \qed \\

It is important to take note of a few things:
\begin{enumerate}[1.]
\item A \emph{finite} direct sum of injective modules is injective because a finite direct sum is equal to a finite direct product. 
\item You can have infinitely many injective modules and have $\bigoplus E_i$ \emph{not} injective.
\item You can have an infinite family of projective modules $\{P_i\}$ but $\prod P_i$ \emph{not} projective. 
\item Each summand of an injective module is injective. 
\end{enumerate}



\subsection{Divisible Modules and Snake Lemma}

\begin{dfn}[Divisible Group]
An abelian group $D$ is divisible if for all $y \in D$ and $n \in \Z \setminus \{0\}$, there is $x \in D$ such that $y=nx$. 
\end{dfn}

\begin{ex}
$_\Z \Q$ is divisible since if $y=a/b$, where $a,b \in \Z$ and $b \neq 0$, we let $x=\frac{a}{bn}$ and $nx=y$.
\end{ex}

\begin{prop}
An abelian group $D$ is divisible if and only if $_\Z D$ is injective. 
\end{prop}

Proof: Assume that $D$ is divisible. Let $I \neq 0$ be a left ideal of $\Z$. So $I=\langle n \rangle$ for some $n$. Consider
\[
\begin{tikzcd}
0 \arrow{r} & \langle n \rangle \arrow[hook]{r} \arrow{d}{f} & \Z \arrow[dotted]{dl}{\exists g} \\
& D & 
\end{tikzcd}
\]
to show that $f$ can be extended, let $y=f(n)$. Let $x \in D$ such that $nx=y$. Let $g: \Z \rightarrow D$, where $g(m)=mx$ and so $g(1)=x$. Then $g(an)=anx=ay$ so that $g|_{\langle x \rangle}=f$. 

Now let $D$ be an injective $\Z$-module. Let $y \in D$ and $0 \neq n \in \Z$. We look at
\[
\begin{tikzcd}
0 \arrow{r} & \langle n \rangle \arrow[hook]{r} \arrow{d}{f} & \Z \arrow[dotted]{dl}{\exists g} \\
& D & 
\end{tikzcd}
\]
Let $f(an) \defeq ay$. So $f$ is a homomorphism of $R$-modules. By Baer's Criterion, there is a $g$ with $g|_{\langle n range}=f$ since $D$ is injective. Let $x=g(1)$. Then $g(n)=n \cdot g(1)=nx$. But $g(n)=f(n)=y$ so that $D$ is divisible. This also shows $_\Z \Q$ is an injective module over $\Z$. \qed \\

It is also important to note that a direct sums of divisible modules is divisible and a quotient of divisible modules is divisible. Our goal to show that for any ring $R$ and $M$ an $R$-module then there is an injective $R$-module $E$ and a monomorphism $M \longrightarrow E$. 

Now assume we have the following diagram with exact rows
\[
\begin{tikzcd}
0 \arrow{r} & A \arrow{r}{f} \arrow{d}{f} & B \arrow{r} \arrow{d}{g} & C \arrow{d}{h} \arrow{r} & 0 \\
0 \arrow{r} & A' \arrow{r} & B' \arrow{r} & C' \arrow{r} & 0
\end{tikzcd}
\]
then there exists an exact sequence
\[
\underbrace{0 \longrightarrow \ker f \longrightarrow \ker g}_{\text{Unv. Property Kernel}} \longrightarrow \underbrace{\ker h}_{\text{Exactness}} \ma{\delta} \underbrace{\coker f \longrightarrow \coker g \longrightarrow \coker h \longrightarrow 0}_{\text{Unv. Property of Cokernel}}
\]
Consider
\[
\begin{tikzcd}
\; & 0 \arrow{d} & 0 \arrow{d} & 0 \arrow{d} & \\
0 \arrow[dotted]{r} & \ker f \arrow{d} \arrow[dotted]{r} & \ker g \arrow{d} \arrow[dotted]{r} & \ker h \arrow{d} & \\
0 \arrow{r} &  A \arrow{d} \arrow{r} & B \arrow{d} \arrow{r} & C \arrow{d} \arrow{r} & 0 \\
0 \arrow{r} & A' \arrow{d} \arrow{r} & B' \arrow{d} \arrow{r} & C' \arrow{r} \arrow{d} & 0 \\
 & \coker f \arrow[dotted]{r} \arrow{d} & \coker g \arrow[dotted]{r} \arrow{d} & \coker h \arrow[dotted]{r} \arrow{d} & 0 \\
 & 0 & 0 & 0 & 
\end{tikzcd}
\]
This important result is often referred to as the Snake Lemma. To see why, simply look at the following diagram:
\[
  \xymatrix@!R{
                & 0                  & 0                  & 0                      \\
0 \ar@{-->}[r]  & {\ker(a)} \ar[r]   & {\ker(b)} \ar[r]   & {\ker(c)}
                \ar@{-}`r[d]`[d]^\delta[d] % curved arrow 1
                                                                               & \\
0 \ar@{-->}[r]  & A    \ar[r]^{f}    & B  \ar[r]^{g}      & C \ar[r]
                 \ar@{}+<0.6cm,0cm>="p1"  %intersection point 1
                                                                               & 0 \\
0 \ar[r]        & A\pp \ar[r]^{f'}
                 \ar@{}[l]+<1cm,0cm>="p2" %intersection point 2
                 \ar`_l[l]+<1.03cm,0cm>`[d]`[d][d] % curved arrow 2
                               & B\pp \ar[r]^{g'}   & C\pp \ar@{-->}[r]        & 0 \\
          & {\coker(a)} \ar[r] & {\coker(b)} \ar[r] & {\coker(c)} \ar@{-->}[r] & 0 \\
          & 0                  & 0                  & 0                            \\
% vertical arrows
\ar"1,2";"2,2"   \ar"1,3";"2,3"     \ar"1,4";"2,4"
\ar"2,2";"3,2"   \ar"2,3";"3,3"     \ar"2,4";"3,4"
\ar"3,2";"4,2"^a \ar"3,3";"4,3"^<<b \ar"3,4";"4,4"^c
\ar"4,2";"5,2"   \ar"4,3";"5,3"     \ar"4,4";"5,4"
\ar"5,2";"6,2"   \ar"5,3";"6,3"     \ar"5,4";"6,4"
% diagonal arrow, with 1 hole
\ar@{-}"p1";"p2"|!{"2,3";"3,3"}\hole
}
\]

\begin{thm}
Let $_\Z D$ be a divisible $\Z$-module and $R$ be a ring. Then $\Hom_\Z(R,D)$ is an injective $R$-module. 
\end{thm}


































