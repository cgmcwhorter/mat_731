% !TEX root = ../../rings_mods.tex

\newpage
\section{Tensor Products and Flat Modules}

\subsection{Tensor Product}

\begin{dfn}[Biadditve $R$-function]
Let $R$ be a ring. Let $A_R$ and $_R B$ be right and left $R$-modules, respectively. Let $G$ be a $\Z$-module. Then a biadditive $R$-function is a function $A \times B \rightarrow G$ with
\begin{enumerate}[1.]
\item
\[
\begin{split}
f(a_1+a_2,b)&=f(a_1,b)+f(a_2,b) \\
f(a,b_1+b_2)&=f(a,b_1)+f(a,b_2)
\end{split}
\]
\item $f(ar,b)=f(a,rb)$
\end{enumerate}
for all $a,a_1,a_2 \in A$ and $b,b_1,b_2 \in B$. If $R$ is commutative, then the function $f$ is just a bilinear function. 
\end{dfn}

\begin{dfn}[Tensor Product]
A tensor product over $R$ of $A_R$ and $_R B$ is an abelian group $A \otimes_R B$ together with a biadditive function $A \times B \ma{f} A \otimes_R B$ satisfying the following universal property: for all $g: A \times B \rightarrow G$, a biadditive homomorphism of abelian groups, there is a unique homomorphism $h: A \otimes_R B \rightarrow G$ commuting the following diagram
\[
\begin{tikzcd}
A \times B \arrow{r}{f} \arrow{d}{g} & A \otimes_R B \arrow[dotted]{dl}{\exists! h} \\
G & 
\end{tikzcd}
\]
\end{dfn}

\begin{thm}
Given $A_R,_R B$, their tensor product exists and is unique up to isomorphism.
\end{thm}

Proof: First, we prove the existence of such an object. Let $F$ be the free abelian group with basis $A \times B$. So the elements of $F$ are finite linear combinations of ordered pairs in $A \times B$ together with integer coefficients. Let $S$ be the subgroup of $F$ generated by elements of the form
\begin{enumerate}[(i)]
\item $(a_1+a_2,b)-(a_1,b)-(a_2,b)$
\item $(a,b_1+b_2)-(a,b_1)-(a,b_2)$
\item $(ar,b)-(a,rb)$
\end{enumerate}

Define $A \otimes_R B \defeq F/S$. Furthermore, define $a \otimes b$ to be the coset of $(a,b)$, i.e. $(a,b)+S$. Observe that for all $a,a_1,a_2 \in A$, $b,b_1,b_2 \in B$, and $r \in R$, we have
\begin{enumerate}[(i)]
\item $(a_1+a_2) \otimes b=a_1\otimes b+a_2 \otimes b$
\item $a\otimes (b_1+b_2)=a\otimes b_1+a\otimes b_2$
\item $ar \otimes b=a \otimes rb$
\end{enumerate}

We have a map $A \times B \ma{f} A \otimes_R B$. It is simple to show that $f$ is biadditive. We also have an exact sequence of $\Z$-modules:
\[
0 \longrightarrow S \stackrel{\text{incl}}{\hooklongrightarrow} F \ma{\overline{f}} A \otimes_R B \longrightarrow 0
\]
We need show that the universal property of tensor products is satisfied.
\[
\begin{tikzcd}
 \;  &  \;  & A \times B \arrow{dr}{f} \arrow[hookrightarrow]{d} &   &   \\
0 \arrow{r} & S \arrow{r} & F \arrow{r} \arrow[shift right=1ex]{d}{g} \arrow[dotted, shift left=1ex]{d}{\exists! \overline{g}} & A \otimes B \arrow{r} \arrow[dotted]{dl}{\exists! h} & 0 \\
   &    & G &   &   \\
\end{tikzcd}
\]
Now $g$ is a biadditive map and $G$ is an abelian group. Now $F$ is free on $A \times B$ so that there exists a unique map $g: F \rightarrow G$ such that $\overline{g}|_{A\times B}=g$. Now as $g$ is biadditive, we have $S \subseteq \ker \overline{g}$. Using the universal property of the cokernel, there exists a unique map $h: A \otimes B \rightarrow G$ with $h\overline{f}=\overline{g}$. Now comparing this with $A \times B \hookrightarrow F$, we get $hf=g$. 

Now we demonstrate uniqueness of the tensor product. Assume that $G_1,G_2$ are tensor products of $A,B$. Then
\[
\begin{tikzcd}
A \times B \arrow{r}{f_1} \arrow{d}{f_2} & G_1 \arrow[dotted, yshift=0.7ex]{dl}{\exists! h_1} \\
G_2 \arrow[dotted,yshift=-0.8ex]{ur}{\exists! h_2} &
\end{tikzcd}
\]
Now there exists a unique map $h_1$ since $G_1$ is a tensor product with $h_1f_1=f_2$. But as $G_2$ is a tensor product, there exists a unique map $h_2$ with $h_2f_2=f_1$. But then $(h_2h_1)f_1=f_1$. Note that the map $1_{G_1}f_1=f_1$ works and must be unique. But then $h_2h_1=1_{G_1}$. Similarly, $h_1h_2=1_{G_2}$ so that we must have $G_1 \cong G_2$. \qed \\

\begin{rem}
The elements of $A \otimes_R B$ are of the form $\sum_{i \in \mathcal{I}} a_i \otimes b_i$, where $a_i \in A$ and $b_i \in B$. These elements are usually \emph{not} of the form $a \otimes b$. 
\end{rem}

\begin{rem}
Assume $R$ is a commutative ring. Let $A,B$ be two $R$-modules. Form the tensor product $A \otimes_R B$ with $F$ and $S$ as given before. Then $A \otimes_R B$ is an $R$-module. 
\[
r\left( \sum_{i \in \mathcal{I}} a_i \otimes b_i \right)=\sum_{i \in \mathcal{I}} ra_i \otimes b_i=\sum_{i \in \mathcal{I}} a_i \otimes rb_i
\]
\end{rem}

Later, we will prove the following:

\begin{thm}
Let $R$ be a commutative ring. Let $A$ be a free module with basis $\{e_i\}_{i \in \mathcal{I}}$ and $B$ a free module with basis $\{f_j\}_{j \in \mathcal{J}}$. Then $A \otimes_R B$ is a free $R$-module with basis $\{e_i\otimes f_j \}_{i\in\mathcal{I},j\in\mathcal{J}}$.
\end{thm}

\begin{cor}
Let $F$ be a field and $V$ a vector space with basis $\{e_i\}_{i \in \mathcal{I}}$ and $W$ a vector space with basis $\{f_j\}_{j \in \mathcal{J}}$, then $V \otimes_F W$ is a vector space with basis $\{e_i\otimes f_j \}_{i\in\mathcal{I},j\in\mathcal{J}}$. If we have $\dim V=n<\infty$, $\dim W=m<\infty$, then $\dim V\otimes_F W=nm$.
\end{cor}

\begin{ex}
Let $F$ be a field. Let $V$ be a 2-dimensional vector space with basis $\{v_1,v_2\}$. We look at $V \otimes_F V$. We know $\dim V\otimes_F V=4$ and a basis for this vector space is $\{v_1\otimes v_1,v_1 \otimes v_2,v_2 \otimes v_1,v_2\otimes v_2\}$. We look at $v_1\otimes v_1+v_2\otimes v_2 \in V \otimes_F V$. We claim we cannot write $v_1\otimes v_1+v_2\otimes v_2$ as $u \otimes w$ for some $u \in V$, $w \in V$. 

Assume to the contrary that this is possible. Let $u=a_1v_1+a_2v_2$ and $w=b_1v_1+b_2v_2$. Then
\[
\begin{split}
v_1 \otimes v_1+v_2\otimes v_2&=(a_1v_1+a_2v_2) \otimes (b_1v_1+b_2v_2) \\
&=a_1b_1v_1 \otimes v_1+a_2b_2v_2\otimes v_2 +a_1b_2v_1\otimes v_2+a_2b_1v_2 \otimes v_1
\end{split}
\]
But then using the chosen basis for the tensor product, we must have
\[
\begin{split}
a_1b_1&=1 \\
a_2b_2&=1 \\
a_1b_2&=0 \\
a_2b_1&=0
\end{split}
\]
So $a_i \neq 0 \neq b_j$ for all $i,j$. This method works for any dimension, not just dimension 2. Only obtains \emph{only} $u\otimes v$'s alone in dimension 1. 
\end{ex}

\begin{ex}
$\Q\otimes_\Z \Z/n\Z$ for $n>1$. The tensor product $\Q \otimes_\Z \Z/n\Z$ is generated (not equal to but generated by) elements of the form $q \otimes \overline{a}$, where $q \in \Q$ and $\overline{a} \in \Z/n\Z$. We claim that $q \otimes \overline{a}=0$ for all $q \in \Q$ and $\overline{a} \in \Z/n\Z$. Let $1=\alpha/\beta$, then
\[
q\otimes_\Z \overline{a} = \frac{\alpha}{n} \otimes_\Z \overline{a}=\frac{n\alpha}{n\beta} \otimes_\Z \overline{a}=\frac{\alpha}{n\beta} \otimes_\Z n \overline{a}= \frac{\alpha}{n\beta} \otimes_\Z 0=0
\]
However, one can't always ``move things over". Take $A \otimes_R B$. Suppose $a \otimes_R b$ and $b=0$ for some $r \in R$ and $a=a'r$, where $a'$ is not necessarily in $A$. Then one \emph{cannot} write
\[
a \otimes_R b \neq a' \otimes_R rb=0
\]
One is reminded of ideals, for $a=a'r$ with $a' \notin A$. 
\end{ex}

\begin{prop}
Let $M$ be an $R$-module. Then $R \otimes_R M \cong M$. 
\end{prop}

Proof: We know $R \otimes_R M$ is a left $R$-module via
\[
r \left(\sum_{i \in \mathcal{I}} r_i \otimes m_i\right)=\sum_{i \in \mathcal{I}} rr_i \otimes m_i
\]
Let $\sum r_i \otimes m_i \in R \otimes_R M$. As $r_i=1\cdot r_i$ and $r_i \otimes m_i=1 \otimes r_im_i$, we have
\[
\sum 1\otimes r_im_i=1\otimes \sum r_im_i
\]
But $\sum r_i m_i \in M$ as $M$ is an $R$-module. Define $g(r,m) \defeq rm$. Now $g$ is biadditive (as an $M$ module). We have $f(r,m)=r \otimes m$.
\[
\begin{tikzcd}
R \times M \arrow{r}{g} \arrow{d}{f} & M \\
R\otimes_R M \arrow[dotted]{ur}{\exists!h}
\end{tikzcd}
\]
Then there exists a unique $h: R \otimes_R M \rightarrow M$ such that the above diagram commutes. We have $h(r \otimes m)=rm$. Now $h$ is a left homomorphism of modules as $sh(r\otimes m)=s(rm)=(sr)m=h(sr\otimes m)$. We need an inverse homomorphism $\overline{h}$. Let $\overline{h}: M \rightarrow R \otimes_R M$ be $\overline{h}(m)=1 \otimes m$. Then
\[
\begin{split}
\overline{h}(rm)&=1\otimes rm \\
&=r \otimes m \\
&=r(1\otimes m) \\
&=r\overline{h}(m)
\end{split}
\]
so that $\overline{h}$ is a homomorphism. Observe also that $h\overline{h}=\overline{h}h=1$. \qed \\

\begin{prop}
If $M_R$ is an $R$-module, then there exists an isomorphism of right $R$-modules, $M \otimes_R R \stackrel{\sim}{\rightarrow} M$. 
\end{prop}

\begin{prop}
If $R$ is a commutative ring and $A,B$ are $R$-modules then there is an isomorphism of $R$-modules $A \otimes_R B \stackrel{\sim}{\rightarrow} B \otimes_R A$. 
\end{prop}

\begin{prop}
Let $I$ be a two-sided ideal of $R$ ($I \lhd R$). Let $_R M$ be a left $R$-module, then $R/I \otimes M$ is a left $R$-module. In fact, it is a left $R/I$-module via 
\[
\begin{split}
R:& r(\overline{s} \otimes m) \defeq r\overline{s} \otimes m \\
R/I:& \overline{r}(\overline{s} \otimes m) \defeq \overline{rs} \otimes m
\end{split}
\]
\end{prop}

\begin{thm}
If $I \lhd R$ and $_R M$, then there exists an isomorphism of $R$-modules (in fact of $R/i$-modules) from $R/I\otimes_R M \stackrel{\sim}{\rightarrow} M/IM$, where
\[
IM=\{\sum \alpha_i m_i \;|\; \alpha_i \in I,m_i\in M\}
\]
\end{thm}

Proof: An element in $R/I \otimes_R M$ is of the form $\sum \overline{r_i} \otimes_R m_i$, where $\overline{r_i} \in R/I$. But
\[
\begin{split}
\sum \overline{r_i} \otimes_R m_i &= \sum (\overline{1} \cdot r_i) \otimes_R m_i \\
&= \sum \overline{1} \otimes_R r_im_i \\
&=1 \otimes m
\end{split}
\]
for some $m \in M$. Now let $R/I \otimes_R M \ma{h} M/IM$ be given by $h(\overline{1} \otimes_R m) \defeq m+IM$. We will denote $m+IM$ as $\overline{m}$. It is clear that $h$ is a homomorphism of left modules. (Why?) Let $\overline{h}: M/IM \rightarrow R/I \otimes_R M$ be given by $\overline{h}(m+IM)=\overline{1} \otimes_R m$. It is clear that $\overline{h}$ is a homomorphism. (Why?) Then $h \overline{h}=\overline{h}h=1$. \qed \\

\begin{ex}
$\Z/m\Z \otimes_\Z \Z/n\Z$. We can take $R=\Z$, $I=m\Z$, and $M=\Z/n\Z$. 
\end{ex}

\subsection{Tensor Products of Homomorphisms} 

Suppose we have $A_R \ma{f} B_R$, $_R A' \ma{f'} _R B'$ be homomorphisms. We claim that there exists an induced homomorphism of abelian groups
\[
A \otimes_R A' \ma{f \otimes f'} B \otimes_R B'
\]
such that $(f \otimes f')(a \otimes a')=f(a) \otimes f'(a')$. We can say that 
\[
\begin{tikzcd}
A \times A' \arrow{r}{g} \arrow{d}{f} & B \otimes B' \\
A \otimes A' \arrow[dotted]{ur}{h} & 
\end{tikzcd}
\]
where $g(a,a')=f(a) \otimes f'(a')$ is a biadditive function. Then the Universal Property of the Kernel says that there exists a unique homomorphism $h: A \otimes A' \longrightarrow B \otimes B'$ with the ``right" property. 

\begin{ex}
If $M_R$ is a right $R$-module and we have a homomorphism of left $R$-modules $_R A \ma{f} _R B$, then we have a homomorphism of abelian groups 
\[
M \otimes_R A \ma{1_M \otimes f} M \otimes_R B
\]
given by $m \otimes a \ma m \otimes f(a)$. 
\end{ex}

Now suppose that we have a short exact sequence
\[
0 \longrightarrow _R A \ma{f} _R B \ma{g} _R C \longrightarrow 0
\]
Given a right $R$-module $M_R$, we can ask about the sequence
\[
0 \longrightarrow M \otimes A \ma{1_M \otimes f} M \otimes B \ma{1_M \otimes g} M \otimes C \longrightarrow 0
\]
This sequence is not \emph{usually} exact. 

\begin{thm}
If $0 \longrightarrow _R A \ma{f} _R B \ma{g} _R C \longrightarrow 0$ is exact and $M_R$ is a right $R$-module, then the sequence $M \otimes A \ma{1_M \otimes f} M \otimes B \ma{1_M \otimes g} M \otimes C \longrightarrow 0$ of abelian groups is exact. Furthermore, if the sequence $0 \longrightarrow A \ma{f} B \ma{g} C \longrightarrow 0$ is split exact, then $0 \longrightarrow M \otimes A \ma{1_M \otimes f} M \otimes B \ma{1_M \otimes g} M \otimes C \longrightarrow 0$ is split exact. 
\end{thm}

Proof: Let $x \in M \otimes_R C$. Now as $g$ is onto
\[
\begin{split}
x&= \sum m_i \otimes c_i \\
&=\sum m_i \otimes g(b_i) \\
&=\sum 1_M(m_i) \otimes g(b_i) \\
&=\sum (1_M \otimes g)(m_i \otimes b_i) \\
&=(1_M \otimes g) \sum m_i \otimes b_i
\end{split}
\]
so that $1_M \otimes g$ is onto. Now
\[
\begin{split}
(1_M \otimes g)(1_M \otimes f)(m \otimes a)&=(1_M \otimes g)(m \otimes f(a)) \\
&=m \otimes \underbrace{g(f(a))}_{=0} =0
\end{split}
\]
as $gf=0$. Therefore, $\im(1_M \otimes f) \subseteq \ker(1_M \otimes g)$. To prove that $\ker(1_M \otimes g) \subseteq \im(1_M \otimes f)$ is much harder and will not be treated here (see page 210 Hungerford \emph{Algebra} or Dummit \& Foote). \qed \\

Let $R$ be a ring. Let $_R M$ be a left $R$-module. Then $\Hom_R(R,M)$ is a left $R$-module via $(rf)(s) \defeq f(sr)$, where $r,s \in R$ and $f \in \Hom_R(R,M)$. We claim that $rf$ is in $\Hom_R(R,M)$. 

\begin{prop}
There is an isomorphism of left $R$-modules $\Hom_R(R,M) \ma{\psi} M$. 
\end{prop}

Proof (Sketch): We have $\psi: f \mapsto f(1)$ a homomorphism. Then let $M \ma{\varphi} \Hom_R(R,M)$ be given by $\varphi(m) \defeq f_M$, where $f_M(r)=rm$. Now $f_M$ is a homomorphism so that $\varphi$ is a homomorphism. Furthermore, $\varphi$ and $\psi$ are inverses of each other. \qed \\

\subsection{Functorial} 

\begin{thm}
Let $R$ be a ring. There is a functorial isomorphism $R \otimes M \ma{\varphi_M} M$ for every left module $M$. 
\end{thm}

Functorial means for all modules $_R M \ma{g} _R N$, the following diagram commutes
\[
\begin{tikzcd}
R \otimes_R M \arrow{r}{\varphi_M} \arrow{d}{1_R \otimes g} & M \arrow{d}{g} \\
R \otimes_R N \arrow{r}{\varphi_N} & N 
\end{tikzcd}
\]
where $\varphi_M,\varphi_N$ are isomorphisms. (Show this) We also have functorial isomorphisms $M \otimes_R R \longrightarrow M$. 

Now let $A_R$ and $_R B$ be $R$-modules with $A'_R,_R B'$ submodules. Let $a \in A'$ and $b \in B'$. Then $a \otimes b \in A' \otimes_R B'$ and $a \otimes b \in A \otimes_R B$. But it can be the case that $a \otimes b=0$ in $A \otimes_R B$ but be nonzero in $A' \otimes_R B'$ so that $A' \otimes_R B'$ is \emph{not} a subgroup of $A \otimes_R B$. 

\begin{ex}
Let $R=\Z$, $A=\Z$, $B=\Z/3\Z$, $A'=3\Z$, and $B'=\Z/3\Z$. Let $0\neq b \in B$, say $b=\overline{1}$. We look at $3 \otimes \overline{1}$. We have $3 \otimes \overline{1} \in A' \otimes B'$ nonzero but inside $A \otimes B$, $A \otimes_R B=\Z \otimes_\Z \Z/3\Z$.
\[
3 \otimes_Z \overline{1}= 1\otimes_\Z \overline{3}=1 \otimes 0=0
\]
\end{ex}

\begin{ex}
We look at $0 \longrightarrow \Z \ma{f} \Z \longrightarrow \Z/3\Z \longrightarrow 0$, where $f(m)=3m$. Let $M=\Z/3\Z$. Then $f \otimes 1_M$ is not injective. We have $f \otimes 1_M: \Z \otimes_\Z \Z/3\Z \longrightarrow \Z \otimes_\Z \Z/3\Z$. 
\[
(f \otimes 1_M)(a \otimes b)=f(a) \otimes b=3a \otimes_\Z b=a \otimes_\Z 3b=a \otimes 0=0
\] 
so $f \otimes 1_M$ is the zero map. This shows that the tensor product of injective maps need not be injective. 
\end{ex}

\begin{thm}[Adjoint Isomorphism Theorem]
Let $R,S$ be rings. Let $A_R, _R B_S, C_S$. Then there is an isomorphism of abelian groups
\[
\Hom_S(A \otimes_R B, C) \ma{\sim} \Hom_R(A, \Hom_S(B,C))
\]
which is functorial in $A,B,$ and $C$.
\end{thm}

We explain what functorality in $A,B,$ and $C$ means. To be functorial in $C$, we have if $g$ is homomorphism $C_S \ma{g} C_S'$, then we have
\[
\begin{tikzcd}
\Hom_S(A \otimes B,C) \arrow{r}{\varphi_{A,B,C}} \arrow{d}{1_R \otimes g} & \Hom_R(A,\Hom_S(B,C)) \arrow{d}{g} \\
\Hom_S(A \otimes B,C') \arrow{r}{\varphi_{A,B,C'}} & \Hom_R(A,\Hom_S(B,C')) 
\end{tikzcd}
\]
where the map on the left is $\Hom(A \otimes B,g)$ and the map on the right is $\Hom(A,\Hom(B,g))$.

Functorality in $A$ means given a homomorphism $A \ma{g} A'$, then $A \otimes B \ma{g \otimes 1_B} A' \otimes B \longrightarrow C$. 
\[
\begin{tikzcd}
\Hom(A' \otimes B,C) \arrow{r}{\varphi_{A',B,C}} \arrow{d}{1_R \otimes g} & \Hom(A',\Hom(B,C)) \arrow{d}{g} \\
\Hom(A \otimes B,C) \arrow{r}{\varphi_{A,B,C}} & \Hom_R(A,\Hom_S(B,C)) 
\end{tikzcd}
\]
where the map on the left is $\Hom(g \otimes 1_B,C)$ and the map on the right is $\Hom(g,\Hom_S(B,C))$ and $\varphi_{A',B,C},\varphi_{A,B,C}$ are isomorphisms. 

Now we show that it means to be functorial in $B$. (Exercise) 

\begin{thm}[Adjoint Isomorphism Theorem]
Let $R,S$ be rings. Let $A_R, _R B_S, C_S$. Then there is an isomorphism of abelian groups
\[
\Hom_S(A \otimes_R B, C) \ma{\sim} \Hom_R(A, \Hom_S(B,C))
\]
which is functorial in $A,B,$ and $C$.
\end{thm}

Proof: We need a map $\varphi: \Hom_S(A \otimes_R B,C) \longrightarrow \Hom_R(A,\Hom_S(B,C))$. Let $f: A \otimes_R B \longrightarrow C$. We have $\varphi(f): A \longrightarrow \Hom_S(B,C)$ and $\varphi(f)(a): B \longrightarrow C$. Now let $\varphi(f)(a)(b) \defeq f(a \otimes b)$. We need to show that $\varphi(f)(a)$ is a $S$-homorphism, $\varphi(f)$ is a $R$-homomorphism, and $\varphi$ is a homomorphism of abelian groups. We need an inverse $\psi: \Hom_R(A,\Hom_S(B,C)) \longrightarrow \Hom_S(A \otimes_R B,C)$. Let $g: A \longrightarrow \Hom_S(B,C)$ be a $R$-module homomorphism. Then $\psi_g$ is a unique homomorphism $A \otimes_R B \longrightarrow C$ such that $\psi_g(a \otimes b)=g(a)(b)$. To demonstrate this, we apply the Universal Property of Tensor Products
\[
\begin{tikzcd}
\; & A \times B \arrow{dl} \arrow{dr} & \\
A \otimes B \arrow{rr}{\exists! \psi_g} & & C 
\end{tikzcd}
\]
It remains to show that $\psi_g$ is a homomorphism of abelian groups and show that $\psi,\varphi$ are inverses of each other. (Exercise) \qed \\

If $_R M$, then there exists a left projective $R$-module $P$ with $P \longrightarrow M \longrightarrow 0$. If $_R M$, then there exists an injective $R$-module $E$ and a monomorphism $M \longrightarrow E$. Furthermore, recall that over $\Z$, a module is injective if and only if it is divisible. So $\Q$ is an injective $\Z$-module. Then we obtain the following result:

\begin{prop}
Every $\Z$-module can be embedded in an injective module. 
\end{prop}

Proof: Let $_\Z M=\bigoplus \Z/L$. Observe $\bigoplus \Z$ is free. As every quotient of a free module is free, $_\Z M$ is free. Now $\Z \subsetneq \Q$ is a submodule so $\bigoplus \Z \subsetneq \bigoplus \Q$. So we have $\bigoplus \Q/L$. This is a quotient of a divisible module so that it is divisible so that it is injective. \qed \\

\begin{prop}
Let $R$ be any ring and let $_\Z D$ be a divisible group. Then the $R$ module $E=\Hom_\Z(_\Z R_R,_\Z D)$ is a left injective $R$-module, where $E$ is a left $R$-module via $(rf)(s)=f(sr)$. 
\end{prop}

Proof: Let $f: A \rightarrow E$. We need to show $f$ can be extended to $B$.
\[
\begin{tikzcd}
0 \arrow{r} & A \arrow{d}{f} \arrow{r}{i} & B \arrow[dotted]{dl} \\
& E &
\end{tikzcd}
\]
That is, we have to prove that $\Hom_R(B,E) \ma{\Hom_R(i,E)} \Hom_R(A,E) \longrightarrow 0$ or showing $\Hom_R(B,\Hom_\Z(R,D)) \longrightarrow \Hom_R(A,\Hom_\Z(R,D)) \longrightarrow 0$. To show this, one need only follow the diagram arrows and get isomorphism compositions with onto maps to get the middle map onto and then perform the same to the top row. (Exercise) 
\[
\begin{tikzcd}
\Hom_R(B,\Hom_\Z(R,D)) \arrow{r} & \Hom_R(A,\Hom_\Z(R,D)) \arrow{r} & 0 \\
\Hom_\Z(R \otimes_R B,D) \arrow{u} \arrow{r} & \Hom_R(R \otimes_R A, D) \arrow{u} \arrow{r} & 0 \\
\Hom_\Z(B,D) \arrow{u} \arrow{r} & \Hom_\Z(A,D) \arrow{u} \arrow{r} & 0 
\end{tikzcd}
\]

\begin{ex}
If $M \ma{h} N$ is an isomorphism, then $\Hom(M,X) \ma{\sim} \Hom(N,X)$. (Exercise) 
\end{ex}

\begin{ex}
If $M \cong N$ and $A$ is another $R$-module, then
\[
\Hom(M,A) \cong \Hom(N,A)
\]
and
\[
\Hom(A,M) \cong \Hom(A,N)
\]
\end{ex}

It is also useful to take note of the following theorem,

\begin{thm}
Let $M_1 \ma{f} M_2 \ma{g} M_3$ be a sequence of $R$-modules. Then the following are equivalent:
\begin{enumerate}[(i)]
\item $M_1 \ma{f} M_2 \ma{g} M_3\longrightarrow 0$ is exact.
\item $g \circ f=0$ and for all $h: M \rightarrow X$ with $hf=0$, there exists a unique $h':M_3 \rightarrow X$ with $h'g=h$ ($g$ is the cokernel of $f$ so this gives us exactness). 
\item For all $R$-modules $X$, we have an induced exact sequence of abelian groups
\[
0 \longrightarrow \Hom(M_3,X) \ma{\Hom(g,X)} \Hom(M_2,X) \ma{\Hom(f,X)} \Hom(M_1,X)
\] 
\end{enumerate}
\end{thm}


\newpage
\subsection{Flat Modules} 

\begin{thm}
\begin{enumerate}[(i)]
\item Let $\{A_i\}_{i \in \mathcal{I}}$ be a family of right $R$-modules and let $B$ be a left $R$-module. Then there exists an isomorphism of abelian groups
\[
\bigoplus_{i \in \mathcal{I}} A_i \otimes_R B \ma{\sim} \bigoplus_{i \in \mathcal{I}} \bigg( A_i \otimes_R B \bigg)
\]
\item Let $A$ be a right $R$-module and let $\{B_i\}_{i \in \mathcal{I}}$ be a family of left $R$-modules. Then there exists an isomorphism of abelian groups
\[
A \otimes_R \left( \bigoplus_{i \in \mathcal{I}} B_i \right) \ma{\sim} \bigoplus_{i \in \mathcal{I}} (A \otimes_R B_i)
\]
\end{enumerate}
Moreover if $R$ is commutative, these are $R$-module isomorphisms. 
\end{thm}

Assume that we have $_S A_R, _R B$ then $A \otimes_R B$ is also a left $S$-module via $s(a \otimes b) \defeq sa \otimes b$. If $A_R,_R B_S \rightarrow A \otimes_R B$ is a right $S$-module. Recall that if $0 \longrightarrow A \ma{f} B \ma{g} C \longrightarrow 0$ is a short exact sequence of left $R$-modules then for all $M_R$, we have the exact sequence
\[
M \otimes A \ma{\text{id}_M \otimes f} M \otimes B \ma{\text{id}_M \otimes g} M \otimes C \longrightarrow 0
\]

\begin{ex}
If $0 \longrightarrow A \ma{f} B \ma{g} C \longrightarrow 0$ splits then $1_M \otimes f$ is also injective. 
\end{ex}

\begin{dfn}[Flat Module]
A right $R$-module $M$ is called flat if for every short exact sequence of left modules
\[
0 \longrightarrow A \ma{f} B \ma{g} C \longrightarrow 0
\]
the induced sequence 
\[
0 \longrightarrow M \otimes A \ma{1_M \otimes f} M \otimes B \ma{1_M \otimes g} M \otimes C \longrightarrow 0
\]
is exact. That is, tonsuring with flat modules preserves exactness. 
\end{dfn}

\begin{ex}
$R_R$ is flat. To see this let $A \ma{f} B$ be a monomorphism. Then
\[
\begin{tikzcd}
R \otimes_R A \arrow{d}{\varphi_A} \arrow{r}{1_R \otimes f} & R \otimes_R B \arrow{d}{\varphi_B} \\
A \arrow{r}{f} & B 
\end{tikzcd}
\]
where the left map is given by $r \otimes a \mapsto ra$ and the right map is given by $r \otimes b \mapsto rb$. Going around the left and bottom of the diagram is a homomorphism and going around the other away is a monomorphism so that $1_R \otimes f$ is a monomorphism. 
\end{ex}

\begin{prop}
Let $\{M_i\}_{i \in \mathcal{I}}$ be a family of right $R$-modules, then $\bigoplus_{i \in \mathcal{I}} M_i$ is flat if and only if $M_i$ is flat for $i \in \mathcal{I}$. 
\end{prop}

Proof (Sketch): The forward direction is an exercise. For the reverse direction, assume that each $M_i$ is flat. We look at $0 \longrightarrow _R A \ma{f} _R B$. But then $M_i \otimes_R A \ma{1_M \otimes f} M_i \otimes_R B$ is a monomorphism for all $i \in \mathcal{I}$. Then we use the fact that $X_i \ma{h_i} Y_i$ is a monomorphism for all $i$ if and only if $\bigoplus X_i \ma{\oplus h_i} \bigoplus Y_i$ is a monomorphism: $h((x_i)) \rightarrow (h_i(x_i))_i$. 
\[
\begin{tikzcd}
\bigoplus X_i \arrow{r} & \bigoplus Y_i \\
X_i \arrow[hook]{u}{k_i^X} \arrow{r}{h_i} & Y_i \arrow{u}{k_i^Y} \\
0 \arrow{u} & 0 \arrow{u}
\end{tikzcd}
\]
\qed \\

\begin{cor}
$M=L \bigoplus K$ is flat if and only if $L$ and $K$ are flat. 
\end{cor}

\begin{cor}
\begin{enumerate}[(i)]
\item Every free module is flat.
\item Every projective module if flat. 
\end{enumerate}
\end{cor}

Proof:
\begin{enumerate}[(i)]
\item We know that a module is free if and only if it is isomorphic to a direct sum of copies of $R$. 
\item We know that if $P$ is projective then there exists a free module $F$ such that $F=P \oplus Q$ for some $Q$. Then $P$ is flat using the previous part. 
\end{enumerate}
\qed \\

It is important to note
\[
\text{Free Modules} \longrightarrow \text{Projective Modules} \longrightarrow \text{Flat Modules} 
\]

We also have the following:

\begin{thm}
If $R$ is a PID and $P$ is projective then $P$ is free. 
\end{thm}

\begin{thm}
If $R$ is Noetherian and $M$ is finitely generated then $M$ is flat if and only if $M$ is projective. 
\end{thm}

\begin{ex}
$_\Z \Q$ is flat but not projective. 
\end{ex}

\subsection{Rings \& Modules of Fractions}

Here let $R$ be a commutative ring. A subset $S$ of $R$ is multiplicative if $1_R \in S$, $0 \notin S$, and $S$ is closed under multiplication. 

\begin{ex}
If $R$ is an integral domain, then $S=R-\{0\}$ is multiplicative. 
\end{ex}

\begin{ex}
If $P \unlhd R$ is a prime ideal of $R$, then $R\setminus P$ is multiplicative. 
\end{ex}

Now let $S \subset R$ be multiplicative. We look at $R \times S$ and introduce a equivalence relation to this set. Let $(a,s) \sim (b,t)$ if and only if there is a $u \in S$ with $u(at-bs)=0$. Let $R_S=R \times S /\sim$. We can make $R_S$ into a commutative ring with unity. Let $a/s$ be equivalent to the class $(a,s)$. We can now define an addition and multiplication to the set $R_S$:
\[
\begin{split}
+&: \frac{a}{s}+\frac{b}{t} \defeq \frac{at+bs}{st} \\
\cdot &: \frac{a}{s}\cdot \frac{b}{t} \defeq \frac{ab}{st} 
\end{split}
\]
One need show that these defined operations are well defined, associative, distributive, and that the addition is commutative. We let 0 denote $0/s$ and 1 by $1/1$. We have a ring homomorphism $R \rightarrow R_S$ given by $\varphi(r)=r/1$. We call $R_S$ the ring of fractions of $S$. 

We can perform the same construction for modules. Let $M$ be an $R$-module. Introduce $M \times S$ with the equivalence relation $(m,s) \sim (m',s')$ if and only if there is a $t \in S$ such that $t(s'm-sm')=0$. Let $m/s$ denote the equivalence class $(m,s)$. We obtain an abelian group $M_S$, $M_S=M \times S/\sim$, with operations
\[
\begin{split}
+&: \frac{m}{s}+\frac{m'}{s'} \defeq \frac{s'm+sm'}{ss'} \\
\cdot &: \frac{r}{t} \cdot \frac{m}{s} \defeq \frac{rm}{ts} 
\end{split}
\]
Observe that if $S \subset R$ is multiplicative nd $_R M \ma{f} _R N$ is a homomorphism, we can let $f_S: M _S \rightarrow N_S$ be given by $f_S(m/s) \defeq \frac{f(m)}{s}$. It is easy to see that $f_S$ is a homomorphism of $R_S$-modules. 

\begin{prop}
Let $0 \longrightarrow A \ma{f} B \ma{g} C \longrightarrow  0$ be a short exact sequence of $R$-modules and let $S \subset R$ be multiplicative. Then the sequence 
\[
0 \longrightarrow A_S \ma{f_S} B_S \ma{g_S} C_S \longrightarrow 0
\]
is a short exact sequence of $R_S$-modules. 
\end{prop}

Proof: To see that $f_S$ is a monomorphism, let $f_S(a/s)=0$. Then $f(a)/s=0=0/1$. There exists a $t \in S$ with $t(f(a)-0)=0$ so $tf(a)=0$. Now as $f$ is a homomorphism so $f(ta)=0$. Now $f$ is injective so $ta=0$. 
\[
\frac{a}{s}=\frac{ta}{ts}=\frac{0}{ts}=0
\]
so that $f_S$ is injective. To see that $g_S$ is onto, let $c/s \in C_S$, where $c \in C$ and $s \in S$. Now $c=g(b)$ for some $b \in B$. 
\[
\frac{c}{s}=\frac{g(b)}{s}=g_S(b/s)
\]
so that $g_S$ is onto. To see exactness at $B_S$,
\[
g_Sf_S(a/s)=g_S(f(a)/s)=g(f(a))/s=0/s=0
\]
Now let $b/s \in \ker g_S$ so that $g_S(0)=g_S(b/s)=g(b)/s$. So there exists a $t \in S$ with $tg(b)=0$. Then we have $g(bt)=0$ so that $bt=f(a)$ for some $a$. But then $b/s=bt/st=f(a)/st=f_S(a/st)$. Therefore, $\ker g_S \subseteq \im f_S$. But by above we also have $\im f_S \subseteq \ker g_S$. \qed \\

It is important to note that taking fractions preserves exact sequences, as the above shows.

\begin{thm}
Let $R$ be commutative. Let $S \subset R$ be multiplicative. Then there exists a natural (functorial) isomorphism of $R_S$-modules, $R_S \otimes_R M \ma{\varphi_M} M_S$.
\end{thm}

Proof: Exercise 

A consequence of this theorem is that if $S \subset R$ is multiplicatively closed, then $R_S$ is a flat $R$-module. 










































